\documentclass{beamer}
\usepackage[utf8]{inputenc}
\usepackage{amsmath}
\usepackage{graphicx}

\title{Tiempo de Respuesta en la Evaluación de un Software}
\author{Phocco Soncco Yeni Rosmirian}
\date{\today}

\begin{document}

\frame{\titlepage}

\begin{frame}
\frametitle{Introducción}
El tiempo de respuesta en la evaluación de un software es un aspecto crítico que puede influir en el éxito del proyecto. Esta presentación examina los factores que afectan el tiempo de respuesta, métodos de evaluación y ejemplos prácticos.
\end{frame}

\begin{frame}
\frametitle{Objetivos de la Evaluación de Software}
\begin{itemize}
    \item Asegurar que el software cumple con los requisitos especificados.
    \item Identificar y corregir defectos.
    \item Evaluar el rendimiento y la escalabilidad.
    \item Verificar la seguridad del software.
    \item Mejorar la usabilidad y la experiencia del usuario.
\end{itemize}
\end{frame}

\begin{frame}
\frametitle{Factores que Influyen en el Tiempo de Respuesta}
\begin{itemize}
    \item \textbf{Complejidad del Software:} A mayor complejidad, mayor tiempo de evaluación.
    \item \textbf{Métodos de Evaluación:} Pruebas manuales vs. pruebas automatizadas.
    \item \textbf{Recursos Disponibles:} Cantidad y experiencia del equipo evaluador.
    \item \textbf{Documentación:} Calidad y claridad de la documentación.
    \item \textbf{Infraestructura:} Disponibilidad y configuración de entornos de prueba.
    \item \textbf{Iteraciones y Retroalimentación:} Evaluaciones ágiles y continuas.
\end{itemize}
\end{frame}

\begin{frame}
\frametitle{Métodos de Evaluación}
\begin{block}{Pruebas Funcionales}
Verifican que el software realiza las funciones especificadas correctamente.
\end{block}
\begin{block}{Pruebas de Rendimiento}
Evalúan cómo se comporta el software bajo diferentes cargas y condiciones.
\end{block}
\begin{block}{Pruebas de Seguridad}
Identifican vulnerabilidades y aseguran que el software es resistente a ataques.
\end{block}
\begin{block}{Pruebas de Usabilidad}
Aseguran que el software es intuitivo y fácil de usar para los usuarios finales.
\end{block}
\end{frame}

\begin{frame}
\frametitle{Proceso de Evaluación}
\begin{enumerate}
    \item \textbf{Planificación:} Definición de objetivos, alcance y métodos de evaluación.
    \item \textbf{Diseño de Pruebas:} Creación de casos de prueba basados en los requisitos del software.
    \item \textbf{Ejecución de Pruebas:} Realización de pruebas y recolección de datos.
    \item \textbf{Análisis de Resultados:} Evaluación de los resultados y detección de problemas.
    \item \textbf{Informe:} Documentación de hallazgos y recomendaciones.
\end{enumerate}
\end{frame}

\begin{frame}
\frametitle{Ejemplo Práctico}
Evaluación de un sistema de gestión de inventarios.
\begin{block}{Pasos a Seguir}
    \begin{enumerate}
        \item \textbf{Definir Requisitos:} Identificar requisitos funcionales y no funcionales.
        \item \textbf{Desarrollar Casos de Prueba:} Crear pruebas automatizadas para las funcionalidades clave.
        \item \textbf{Pruebas de Rendimiento:} Realizar pruebas de carga para evaluar la capacidad de respuesta bajo diferentes niveles de uso.
        \item \textbf{Pruebas de Seguridad:} Realizar pruebas de penetración para identificar vulnerabilidades.
        \item \textbf{Feedback de Usuarios:} Recopilar opiniones y experiencias de usuarios finales para mejorar la usabilidad.
    \end{enumerate}
\end{block}
\end{frame}

\begin{frame}
\frametitle{Ejemplo de Caso de Prueba}
\begin{block}{Funcionalidad: Añadir Nuevo Producto}
\begin{itemize}
    \item \textbf{Precondiciones:} El usuario debe estar autenticado y tener permisos de administrador.
    \item \textbf{Acciones:} El usuario navega al formulario de "Añadir Producto", introduce los detalles del producto y guarda.
    \item \textbf{Resultados Esperados:} El producto se añade a la base de datos y aparece en la lista de inventarios.
\end{itemize}
\end{block}
\end{frame}

\begin{frame}
\frametitle{Resultados de la Evaluación}
\begin{itemize}
    \item \textbf{Pruebas Funcionales:} Todas las funcionalidades clave se comportaron según lo esperado.
    \item \textbf{Pruebas de Rendimiento:} El sistema soportó hasta 1000 usuarios simultáneos sin degradación significativa en el rendimiento.
    \item \textbf{Pruebas de Seguridad:} Se identificaron y corrigieron 3 vulnerabilidades críticas.
    \item \textbf{Pruebas de Usabilidad:} Los usuarios encontraron el sistema intuitivo, pero sugirieron mejoras en la navegación.
\end{itemize}
\end{frame}

\begin{frame}
\frametitle{Conclusiones}
El tiempo de respuesta en la evaluación del software depende de múltiples factores y una planificación adecuada. Identificar y mitigar estos factores puede mejorar significativamente la eficiencia y efectividad de la evaluación.
\end{frame}

\begin{frame}
\frametitle{Bibliografía}
\begin{thebibliography}{9}
    \bibitem{ref1} Sommerville, I. (2011). \textit{Ingeniería de Software}. Pearson Educación.
    \bibitem{ref2} Pressman, R. S. (2014). \textit{Ingeniería de Software: Un Enfoque Práctico}. McGraw-Hill.
    \bibitem{ref3} Myers, G. J., Sandler, C., \& Badgett, T. (2011). \textit{The Art of Software Testing}. John Wiley \& Sons.
    \bibitem{ref4} IEEE Standard for Software and System Test Documentation (IEEE Std 829-2008).
\end{thebibliography}
\end{frame}

\end{document}
